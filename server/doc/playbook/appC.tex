\chapter{Description of Gods}
\label{app:gods}
\index{gods, description}

Below in boxes, the gods in your compiled version of \cf\ are shown.
Use the command {\tt wograld -m8} to check if the information
presented here is accurate.
(Note: you need to have compiled with the {\tt DUMP\_SWITCHES} and
{\tt MULTIPLE\_GODS} flags for this to work!) The boxed attributes 
have meaning as follows:
\vskip 12pt
\begin{tabular}{ll} 
Enemy cult: & Name of the enemy god \\
Aligned race(s): & Names of races friendly to the cult. The priest \\
	 & of this cult has greater power over these creatures. \\
	 & In some cases the prayer {\tt summon cult monsters} \\ 
	 & will summon these monsters to help the priest. \\
Enemy race(s): & Names of races hated by the cult. ``Holy word'' \\ 
	 & prayers of this god may be used to kill these \\
	 & creatures. \\
Attacktype(s): & Attacktypes used by this god's avatar and \\
	 & by cult {\tt cause wounds} prayers. \\
Immunity: & Granted by the {\tt holy possession} prayer. \\
Protected: & Granted to a cult priest and by the {\tt bless} \\
	 & prayer. \\
Vulnerable: & Given to a cult priest and by the {\tt curse } \\
	 & prayer. \\
Attunded: & The cult priest is attuned to these spellpaths. \\
Repelled: & The cult priest is repelled to these spellpaths. \\
Denied: & The cult priest is denied use of these spellpaths. \\
Added gifts/limits: & A cult priest has these addtional benefits\\ 
 	& and restrictions. \\
\end{tabular}
\vskip 12pt
Note that not all gods have values for all possible attributes. In
this case, no attribute will appear in that god's box.
 
\begin{longtable}{|p{4cm}p{9cm}|} \hline 
\input{gods.tex}
\end{longtable}

