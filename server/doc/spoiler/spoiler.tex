\documentclass[11pt, a4paper]{article}
\usepackage{epsfig}
\usepackage[latin1]{inputenc}
\usepackage[T1]{fontenc}
\usepackage[english]{babel}
\usepackage{longtable}
\begin{document}
\title{Wograld in vital numbers}
\author{
\input version
}
\maketitle
\LTchunksize=1000
\setlongtables


\section*{General}
This guide is intended to present the player to his opponents and the
``tools'' of his trade.  The tables in this guide are generated
completely from the wograld source, so you may sometimes see
monsters or items here before they can be encountered in the game.

\subsection*{Enchantments}
Enchanted items are items that is better than the basic type.
They are identified by the {\it +1}, {\it +2}, {\it +3} or {\it +4} at 
the end of the item name.
Also, the higher the number, the rarer the item is.
The enchantments affect the value, weight and effect of the item;
i.e. for armour its {\it ac} (armour class),
for weapons its {\it wc} (weapon class).
Items that already have a magical effect are never enchanted.


\section*{Maxstats}

The following table shows the maximum value the different player
classes can reach in a stat. It also shows how your basic stats will
be changed by choosing a different class. When you roll your
character, the stats displayed are the stats you will get as a human.
When satisfied, you can step through a number of classes, each with
special bonuses in stats.

{\small 
\begin{center}
\begin{tabular}{|c|c|l|l|l|l|l|l|l|p{4cm}|}
\hline
Type& &         Str&    Dex&    Con&    Int&    Wis&    Pow& Cha &Special\\
\hline
\hline
\input stats.tex
\hline
\end{tabular}
\end{center}
}

A barbarian has a maximum strength which is 4 higher than a human --
that means he will begin with an additional 4 points added to his
strength roll. On the other hand, a barbarian can never get above 12
in intelligence.  This means that your rolled character will have 8
less in intelligence if you choose that class.  It also means that you
can't be a barbarian if you roll less than 8 in intelligence -- the
poor barbarian would have had a negative stat.

You can never roll a character with better stats than an average of
straight 15's, and you can't roll higher than 18 in a stat. These
values are the maximum values for your ``natural'' dexterity,
constitution etc. You can raise your natural stats by drinking
potions.

However, there are plenty of items which give you bonuses to your
stats even {\em beyond} your class' limit -- swords, armours and rings to
name the most important. You can also read scrolls or cast spells to
temporarily raise your stats.  The absolute maximum value is 30, and
the player class doesn't matter here.

\section*{Weapons}

\subsection*{Weapons}
Notice that the weight and damage differs on seemingly equal weapons.
We suggest that you wield the {\em identified} weapons,
to choose the better one.

{\small 
\begin{longtable}{|c|c|r|r|r|c|c|r|r|r|}
\hline
Name&&Dam&Speed&Weight&Name&&Dam&Speed&Weight\\
\hline
\hline
\endhead
\hline
\endfoot
\input weap.tex
\end{longtable}
}

Some weapons also have a separate effect:

{\small 
\begin{tabular}{l c p{10cm}}
\input weapmag.tex
\end{tabular}
}
\subsection*{Bows}

The {\em rate} column in the following table shows the relative rate
of fire. E.g. to cock a bow with a ``rate of fire'' of $^1/_2$, you
need only half the time of what is needed with a bow with a ``rate of
fire'' of $^1/_1$.

The damage done by the impact of a bolt fired from a crossbow is
constant.  However, with an ordinary bow you can pull the arrow
further back if you are strong, and it would thus do more damage.

{\small 
\begin{center}
\begin{tabular} {|c|c|r|r|r|}
\hline
Name&&Rate&Dam&Weight\\
\hline
\hline
\input bow.tex
\hline
\end{tabular}
\end{center}
}

\subsection*{Special weapons}

This section shows the different ``Special weapons'' which exist in
wograld.  Although the weapons are supposed to be unique, there may
exist several of them...

{\small 
\begin{longtable}{|c|c|r|r|p{7cm}|}
\hline
Name&&Dam&Max. speed&Special\\
\hline
\hline
\endhead
\hline
\endfoot
\input arche.tex
\end{longtable}
}

\section*{Armour}
 
Armour is essential to surviving in Wograld.
The basic idea is that the less {\it ac} (armour class) you have -- the more 
difficult you are to hit.
The {\it armour} value represents the reduction in physical damage in percent.
There are several types of armour in Wograld.You may only wear {\em one}
of the different main types of armour (except magical armour).

The different main types consist of these :
\subsection*{Body Armour}

{\small 
\begin{center}
\begin{tabular}{|c|c|r|r|r|r|p{4cm}|}
\hline
Type&&Ac& Armour&Weight&Max. speed&Magic\\
\hline
\hline
\input arm.tex
\hline
\end{tabular}
\end{center} 
}
\subsection*{Helmets}

{\small 
\begin{center}
\begin{tabular}{|c|c|r|r|r|p{6cm}|}
\hline
Type&&Ac&Armour&Weight&Magic\\
\hline
\hline
\input helmet.tex
\hline
\end{tabular}
\end{center} 
}
\subsection*{Shields}

{\small 
\begin{center}
\begin{tabular}{|c|c|r|r|r|p{5cm}|}
\hline
Type&&Ac&Armour&Weight&Magic\\
\hline
\hline
\input shield.tex
\hline
\end{tabular}
\end{center} 
}

\subsection*{Other Clothing}

{\small 
\begin{center}
\begin{tabular}{|c|c|r|l|}
\hline
Type&&Armour&Magic\\
\hline
\hline
\input mag.tex
\hline
\end{tabular}
\end{center} 
}

\section*{Magic}

Magic is brought into play by various means.  The only way to actually 
learn the spells, is to read them from a book. Both scrolls and books
will disappear after being read ({\em applied} actually). Magic that
comes from quaffing ({\em applying}) a potion will stay in effect over
a period of time. Naturally not all magic found in wands
would be found in e.g. scrolls etc.{\it Scroll of large fireball} or
{\it Potion of poison} would be ridiculous.

The {\it Wonder} spell will produce random magic (rather unpredictable).

{\small 
\begin{longtable}{|l|c|c|r|r|c|c|c|} 
\hline
Name& & &Level&Sp.&Wands&Scrolls&Books\\
\hline 
\hline
\endhead
\hline
\endfoot
\input spells.tex
\end{longtable} 
}

\section*{Monsters}

The monsters are your opponents in Wograld.  Actually the only way
to gain experience in this game is to bash monsters (or your fellow
players, but you probably won't last long if you choose that route to
``fame'').  The more {\it hitpoints} the monsters have, the longer it
takes to kill the suckers.  Unfortunately, the monsters tend to strike
back...  Thus the stronger the monsters are -- the more damage you
take, and vice versa.

{\small 
\subsection*{The monsters}
\begin{longtable}{|p{2cm}|c|c|r|r|r|p{5cm}|}
\hline
Name&&Gen&Exp&Hp&Ac&Special\\
\hline
\hline
\endhead
\hline
\endfoot
\input monput.tex
\end{longtable}
}
\end{document}
